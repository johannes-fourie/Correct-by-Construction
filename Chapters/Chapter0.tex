% Chapter Template

\chapter{Process and Technique} % Main chapter title

\label{Chapter_ Process_and_Technique} % Change X to a consecutive number; for referencing this chapter elsewhere, use \ref{ChapterX}

%----------------------------------------------------------------------------------------
%	SECTION 1
%----------------------------------------------------------------------------------------

Software development can be seen as having two aspects: 
\begin{itemize}
	\item process and 
	\item technique.
\end{itemize}

Process describes the people involved in developing the software and the 
interactions between them. The process can describe things like team 
structures, work planning, reviewing work done and delivering the software. 

Technique describes technical aspects of software development. Technique can 
describe things like what programming languages and technologies to used, 
how the software is structured and tested, and also how the software is delivered. 

Process and technique can overlap and inform each other as in the case of software
deployment. If we wanted to do continuous delivery we would have to review 
the completed software on an as needed basis and not a fixed schedule (process). 
We might then also structure the software as micro-services and not a monolith 
application (technique).

Some development methodologies focus more on process and others more on technique.

\paragraph*{Agile software development (Agile)}
Agile focuses on delivering useful software in a timely manner by focusing on process.
Agile's only opinion on technique is that the design should be kept simple.

\paragraph*{Correct by Construction (CbyC)}
CbyC uses formal methods as a technique along with a process that 
aligns with the formal methods. Formal methods are mathematical rigorous techniques
used to validate software designs and code.

If we combine Agile process with CbyC technique and process
we can have a methodology that is both rigorous and productive.
