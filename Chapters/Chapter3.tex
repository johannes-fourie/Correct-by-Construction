% Chapter Template

\chapter{Applying the methodology} % Main chapter title

\label{Chapter_Applying_the_methodology} % Change X to a consecutive number; for referencing this chapter elsewhere, use \ref{ChapterX}

%----------------------------------------------------------------------------------------
\section{First iteration}

%----------------------------------------------------------------------------------------
\section{Second iteration}

%----------------------------------------------------------------------------------------
\section{The rest of the life cycle}




Testing is usually the main method of verification and validation. The normal testing
method follows these steps:
\begin{enumerate}
	\item test individual units; 
	\item integrate them and test the integration; 
	\item then test the system as a whole. 
\end{enumerate}
This approach is inefficient because unit testing is ineffective and expensive. 
Unit testing is ineffective because most errors are interface errors, not internal
errors in units. Unit testing is expensive because you have to build test 
harnesses to test units in isolation \parencite{CbyCPraxis}.

A more efficient and effective approach is to incrementally built the system from
the top down. Each build is a real (if small) system and the system can be completely 
exercised in a real environment. This reduces the integration risk \parencite{CbyCPraxis}.

\subsection{C\#}

\subsubsection{Design by contract}
CbyC program design is based on information flow expressed as code contracts 
\parencite{CbyCMan}. C\# does not have built in code contracts any more, but it
is very simple to implement code contracts using standard C\# language features.

\subsubsection{Property based testing}
C\# has very little static analysis tools able to mathematically prove correctness.
As an alternative we will use property based testing to exercise the code and show 
correctness \parencite{QuickCheck} \parencite{Hamlet94randomtesting}. We will be
using the FsCheck framework for our property based tests \parencite{FsCheck_home}.  